%%%%%%%%%%%%%%%%%
% This is an sample CV template created using altacv.cls
% (v1.7, 9 August 2023) written by LianTze Lim (liantze@gmail.com). Compiles with pdfLaTeX, XeLaTeX and LuaLaTeX.
%
%% It may be distributed and/or modified under the
%% conditions of the LaTeX Project Public License, either version 1.3
%% of this license or (at your option) any later version.
%% The latest version of this license is in
%%    http://www.latex-project.org/lppl.txt
%% and version 1.3 or later is part of all distributions of LaTeX
%% version 2003/12/01 or later.
%%%%%%%%%%%%%%%%

%% Use the "normalphoto" option if you want a normal photo instead of cropped to a circle
% \documentclass[10pt,a4paper,normalphoto]{altacv}
\PassOptionsToPackage{dvipsnames}{xcolor}
\documentclass[10pt,a4paper,ragged2e,withhyper]{altacv}
%% AltaCV uses the fontawesome5 and packages.
%% See http://texdoc.net/pkg/fontawesome5 for full list of symbols.

% Change the page layout if you need to
\geometry{left=1cm,right=1cm,top=1.25cm,bottom=1.25cm,columnsep=1.2cm,footskip=2\baselineskip}

% The paracol package lets you typeset columns of text in parallel
\usepackage{paracol}

%Packages
\usepackage[utf8]{inputenc}
\usepackage[T1]{fontenc}
\usepackage{hyperref}

% Change the font if you want to, depending on whether
% you're using pdflatex or xelatex/lualatex
% WHEN COMPILING WITH XELATEX PLEASE USE
% xelatex -shell-escape -output-driver="xdvipdfmx -z 0" sample.tex
\ifxetexorluatex
  % If using xelatex or lualatex:
  \setmainfont{Roboto Slab}
  \setsansfont{Lato}
  \renewcommand{\familydefault}{\sfdefault}
\else
  % If using pdflatex:
  \usepackage[rm]{roboto}
  \usepackage[defaultsans]{lato}
  % \usepackage{sourcesanspro}
  \renewcommand{\familydefault}{\sfdefault}
\fi

% Change the colours if you want to
\definecolor{accent}{HTML}{000e17}
\definecolor{heading}{HTML}{000e17}
\definecolor{emphasis}{HTML}{696969}
\definecolor{body}{HTML}{01415f}
%\colorlet{name}{black}
%\colorlet{tagline}{PastelRed}
%\colorlet{heading}{DarkPastelRed}
%\colorlet{headingrule}{GoldenEarth}
%\colorlet{subheading}{PastelRed}
\colorlet{accent}{accent}
\colorlet{emphasis}{emphasis}
\colorlet{body}{body}


% Change some fonts, if necessary
%\renewcommand{\namefont}{\Huge\rmfamily\bfseries}
%\renewcommand{\personalinfofont}{\footnotesize}
%\renewcommand{\cvsectionfont}{\LARGE\rmfamily\bfseries}
%\renewcommand{\cvsubsectionfont}{\large\bfseries}


% Change the bullets for itemize and rating marker
% for \cvskill if you want to
\renewcommand{\cvItemMarker}{{\small\textbullet}}
\renewcommand{\cvRatingMarker}{\faCircle}
% ...and the markers for the date/location for \cvevent
% \renewcommand{\cvDateMarker}{\faCalendar*[regular]}
% \renewcommand{\cvLocationMarker}{\faMapMarker*}


% If your CV/résumé is in a language other than English,
% then you probably want to change these so that when you
% copy-paste from the PDF or run pdftotext, the location
% and date marker icons for \cvevent will paste as correct
% translations. For example Spanish:
% \renewcommand{\locationname}{Ubicación}
% \renewcommand{\datename}{Fecha}


%% Use (and optionally edit if necessary) this .tex if you
%% want to use an author-year reference style like APA(6)
%% for your publication list
% % When using APA6 if you need more author names to be listed
% because you're e.g. the 12th author, add apamaxprtauth=12
\usepackage[backend=biber,style=apa6,sorting=ydnt]{biblatex}
\defbibheading{pubtype}{\cvsubsection{#1}}
\renewcommand{\bibsetup}{\vspace*{-\baselineskip}}
\AtEveryBibitem{%
  \makebox[\bibhang][l]{\itemmarker}%
  \iffieldundef{doi}{}{\clearfield{url}}%
}
\setlength{\bibitemsep}{0.25\baselineskip}
\setlength{\bibhang}{1.25em}


%% Use (and optionally edit if necessary) this .tex if you
%% want an originally numerical reference style like IEEE
%% for your publication list
\usepackage[backend=biber,style=ieee,sorting=ydnt,defernumbers=true]{biblatex}
%% For removing numbering entirely when using a numeric style
\setlength{\bibhang}{1.25em}
\DeclareFieldFormat{labelnumberwidth}{\makebox[\bibhang][l]{\itemmarker}}
\setlength{\biblabelsep}{0pt}
\defbibheading{pubtype}{\cvsubsection{#1}}
\renewcommand{\bibsetup}{\vspace*{-\baselineskip}}
\AtEveryBibitem{%
  \iffieldundef{doi}{}{\clearfield{url}}%
}


%% sample.bib contains your publications
\addbibresource{sample.bib}


\begin{document}
\name{Tom M. Dobson}
\tagline{Data Consultant}
%% You can add multiple photos on the left or right
%\photoR{2.8cm}{Globe_High}
%\photoL{2.5cm}{Yacht_High,Suitcase_High}

\personalinfo{%
  % Not all of these are required!
  \email{tom.dobson@theinformationlab.co.uk}
%  \phone{+44 (0)7968 674704}
%  \mailaddress{Åddrésş, Street, 00000 Cóuntry}
  \location{London, UK}
%  \homepage{www.homepage.com}
%  \twitter{@twitterhandle}
  \linkedin{linkedin.com/in/tomdobs}
  \github{github.com/dobsontxx}
%  \orcid{0000-0000-0000-0000}
  %% You can add your own arbitrary detail with
  %% \printinfo{symbol}{detail}[optional hyperlink prefix]
  % \printinfo{\faPaw}{Hey ho!}[https://example.com/]
  %% Or you can declare your own field with
  %% \NewInfoFiled{fieldname}{symbol}[optional hyperlink prefix] and use it:
  % \NewInfoField{gitlab}{\faGitlab}[https://gitlab.com/]
  % \gitlab{your_id}
  %%
  %% For services and platforms like Mastodon where there isn't a
  %% straightforward relation between the user ID/nickname and the hyperlink,
  %% you can use \printinfo directly e.g.
  % \printinfo{\faMastodon}{@username@instace}[https://instance.url/@username]
  %% But if you absolutely want to create new dedicated info fields for
  %% such platforms, then use \NewInfoField* with a star:
  % \NewInfoField*{mastodon}{\faMastodon}
  %% then you can use \mastodon, with TWO arguments where the 2nd argument is
  %% the full hyperlink.
  % \mastodon{@username@instance}{https://instance.url/@username}
}


\makecvname
%% Depending on your tastes, you may want to make fonts of itemize environments slightly smaller
% \AtBeginEnvironment{itemize}{\small}

%% Set the left/right column width ratio to 6:4.
\columnratio{0.62}

% Start a 2-column paracol. Both the left and right columns will automatically
% break across pages if things get too long.
\begin{paracol}{2}
\cvsection{Experience}

\cvevent{Data Consultant}{The Information Lab}{April 2022 -- Ongoing}{London}
\begin{itemize}
\item Something
\item Something
\end{itemize}

\divider

\cvevent{Advanced Analytics \& Reporting Analyst}{IAG Cargo}{Feb 2023 -- Aug 2023}{London}
\begin{itemize}
\item Job description 1
\item Job description 2
\end{itemize}

\divider

\cvevent{Tableau/Alteryx Consultant}{BBC}{July 2022 -- Jan 2023}{London}
\begin{itemize}
\item Job description 1
\item Job description 2
\end{itemize}

\cvsection{Projects}

\cvevent{Project 1}{Funding agency/institution}{}{}
\begin{itemize}
\item Details
\end{itemize}

\divider

\cvevent{Project 2}{Funding agency/institution}{Project duration}{}
A short abstract would also work.

\medskip

\cvsection{A Day of My Life}

% Adapted from @Jake's answer from http://tex.stackexchange.com/a/82729/226
% \wheelchart{outer radius}{inner radius}{
% comma-separated list of value/text width/color/detail}
\wheelchart{1.5cm}{0.5cm}{%
  6/8em/accent!30/{Sleep,\\beautiful sleep},
  3/8em/accent!40/Hopeful novelist by night,
  8/8em/accent!60/Daytime job,
  2/10em/accent/Sports and relaxation,
  5/6em/accent!20/Spending time with family
}

% use ONLY \newpage if you want to force a page break for
% ONLY the current column
%\newpage

%% Switch to the right column. This will now automatically move to the second
%% page if the content is too long.
\switchcolumn

\cvsection{Contact}
\makecvinfo

\cvsection{Education}
\cvevent{MSc. (R) Biological Sciences}{University of Bristol}{Sep 2020 -- Aug 2021}{}
\small \faTrophy \;Nominee · Best Postgraduate Thesis
\\\smallskip

\small \faFile* Thesis: Investigating the effects of low-dose UV-B and ambient temperature on dark-induced leaf senescence in Arabidopsis
\\\smallskip
%
% \faTrophy \;Best Postgraduate Thesis [Nominee]
% \faCode \;Python, R, LaTeX
\cvtag{Python}
\cvtag{R}
\cvtag{SPSS}
\cvtag{\LaTeX}

\divider

\cvevent{BSc.\ Biology}{University of Bristol}{Sep 2016 -- June 2019}{}
\small \faTrophy \;First Class Hons.

%%%%%%%%%%%%%%%%%%%%%%%%%%%%%%% Skills %%%%%%%%%%%%%%%%%%%%%%%%%%%%%%%

\cvsection{Skills}
\cvskill{Tableau}{5}
\cvskill{SQL}{4}
\cvskill{Alteryx}{4}
\cvskill{Python}{3}
\cvskill{Power BI}{3}
\small{$\bullet$  = Proficient\hfill 
$\bullet$$\bullet$$\bullet$$\bullet$$\bullet$ = Expert}\\

%%%%%%%%%%%%%%%%%%%%%%%%%%%%%%% Strengths %%%%%%%%%%%%%%%%%%%%%%%%%%%%%%%

\cvsection{Tools \& Tech}
\textbf{\faDatabase Databases}\\
\smallskip
\cvtag{Oracle 11g}
\cvtag{Snowflake}

\divider

\textbf{Management}\\
\smallskip
\cvtag{GitHub}
\cvtag{Jira}
\cvtag{Notion}


\divider

\textbf{SQL Development/Data Engineering}\\
\smallskip
\cvtag{SQL}
\cvtag{Git}
\cvtag{Python}
\cvtag{yaml}

\divider

\textbf{Data Science}\\
\smallskip
\cvtag{Jupyter}
\cvtag{numpy}
\cvtag{pandas}
%\cvtag{PostgreSQL}

\cvsection{Certificates}
\cvevent{\small Tableau Certified Data Analyst}{}{August 2023}{}
%\faCertificate https://www.credly.com/badges/ee7090eb-b41f-48f9-90ad-21d1d45b7966/public\_url

% \divider

\cvsection{Referees}

% \cvref{name}{email}{mailing address}
\cvref{Prof.\ Alpha Beta}{Institute}{a.beta@university.edu}
{Address Line 1\\Address line 2}

\divider

\cvref{Prof.\ Gamma Delta}{Institute}{g.delta@university.edu}
{Address Line 1\\Address line 2}


\end{paracol}


\end{document}
